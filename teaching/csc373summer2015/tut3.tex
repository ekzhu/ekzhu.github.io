%% LyX 2.1.3 created this file.  For more info, see http://www.lyx.org/.
%% Do not edit unless you really know what you are doing.
\documentclass[landscape,english]{slides}
\usepackage[T1]{fontenc}
\usepackage[latin9]{inputenc}
\setcounter{secnumdepth}{1}
\setcounter{tocdepth}{1}
\usepackage{amsmath}

\makeatletter

%%%%%%%%%%%%%%%%%%%%%%%%%%%%%% LyX specific LaTeX commands.
\pdfpageheight\paperheight
\pdfpagewidth\paperwidth

%% Because html converters don't know tabularnewline
\providecommand{\tabularnewline}{\\}

%%%%%%%%%%%%%%%%%%%%%%%%%%%%%% Textclass specific LaTeX commands.
\newenvironment{lyxlist}[1]
{\begin{list}{}
{\settowidth{\labelwidth}{#1}
 \setlength{\leftmargin}{\labelwidth}
 \addtolength{\leftmargin}{\labelsep}
 \renewcommand{\makelabel}[1]{##1\hfil}}}
{\end{list}}

%%%%%%%%%%%%%%%%%%%%%%%%%%%%%% User specified LaTeX commands.
\usepackage{algorithm,algpseudocode}

\makeatother

\usepackage{babel}
\begin{document}
Title

\textbf{CSC373 Tutorial 3, Summer 2015}

TA: Eric Zhu

Problems are based on tutorial exercises from previous offering of
the course by Francois Pitt (Fall 2014).

Intro

Brief review of dynamic programming:
\begin{itemize}
\item Recursive structure of sub-problems
\item Data structure for storing optimal values for sub-problems (why)
\item Iterations to compute optimal values of sub-problems
\item Reconstruction of optimal solution (s)
\end{itemize}
Make change 1

\textbf{Making Change}

Consider the problem of making change when the denominations are arbitrary.
\begin{lyxlist}{00.00.0000}
\item [{Input:}] Positive integer \textquotedbl{}amount\textquotedbl{}
$n$, positive integer \textquotedbl{}denominations\textquotedbl{}
$d=[d_{1},d_{2},...,d_{m}]$ and $d_{1}<d_{2}<...<d_{m}$. 
\item [{Output:}] \textbf{All possible ways} to make change: lists of \textquotedbl{}coins\textquotedbl{}
$c=[c_{1},c_{2},...,c_{k}]$ where each $c_{i}$ is in $d$, repeated
coins are allowed ($c_{i}=c_{j}$ for $i\ne j$), $\sum_{i=1}^{k}c_{i}=n$.
If no solution is possible, $k=0$ and the list is empty.
\end{lyxlist}
Make change 2

\textbf{Recursive Structure}

Finding a recursive structure of sub-problems - we need to divide
a problem into smaller problems, and obtain the optimal solution based
on the optimal solutions of the smaller problems.

make change 3

\textbf{Problem: }find solution giving amount $n$ and denominations
$d=[d_{1},d_{2},...,d_{m}]$: $S(n,[d_{1},...,d_{m}])$

\textbf{Sub-problems: }
\begin{enumerate}
\item Find solutions giving amount $n$ and denominations $d_{1},d_{2},...,d_{m-1}$,
i.e. solve the original problem without using $d_{m}$: $S(n,[d,...,d_{m-1}])$
\item Find solutons giving amount $n-d_{m}$and denominations $d_{1},d_{2},...,d_{m}$,
i.e. use $d_{m}$ once, and solve a smaller problem with the same
denominations: $S(n-d_{m},[d_{1},...,d_{m}])$
\end{enumerate}
make change 4

\textbf{Solution: }take union of the solutions for sub-problems 1
and 2:

$S(n,[d_{1},...,d_{m}])=S(n,[d,...,d_{m-1}])\cup(S(n-d_{m},[d_{1},...,d_{m}])+[d_{m}])$
if $n-d_{m}\ge0$, and

$S(n,[d_{1},...,d_{m}])=S(n,[d,...,d_{m-1}])$ if $n-d_{m}<0$, as
there is no solution to make negative change with positive coins.

For example: 

\begin{align*}
S(5,[1,2,5]) & =S(5,[1,2])\cup(S(0,[1,2,5])+[5])\\
 & =S(5,[1])\cup(S(3,[1,2]))\cup\{[]+[5]\}\\
 & ...\\
 & =\{[1,1,1,1,1]\}\cup\{[1,1,1,2],[1,2,2]\}\cup\{[5]\}
\end{align*}


2-d array:

We need a data structure to store the number of solutions to the sub-problems.
Remember edit distance? 

We can use a table (2-D array) of size $(n+1)\times m$, because there
are 2 input variables ($n$ and $d$) and all their values are discrete,
and we need an additional row to store the sub-probelms involve zero
amounts ($n=0$).

What is the recurrence relation for the values in the table?

2-d array: example 1

Example: $n=5$ and $d=\{1,2,5\}$

\begin{tabular}{|c|c|c|c|}
\hline 
Amount & 1 & 2 & 5\tabularnewline
\hline 
\hline 
0 & 1 &  & \tabularnewline
\hline 
1 &  &  & \tabularnewline
\hline 
2 &  &  & \tabularnewline
\hline 
3 &  &  & \tabularnewline
\hline 
4 &  &  & \tabularnewline
\hline 
5 &  &  & \tabularnewline
\hline 
\end{tabular}

Row $i$ represents amount $i$, and column $j$ represents using
the denominations up to $d_{j}$. Each cell in the table stores the
number of solutions to the sub-problem $S(i,[d_{1},...,d_{j}])$.
For example, cell at row 0, column 1, represented as $(0,1)$, stores
the number of solutions to the sub-problem $S(0,[1])$, and the number
of solutions is 1 - just use 0 coins to make \$0 change. Cell $(1,2)$
is $S(1,[1,2])$, and so on.

2-d array: example 2

\begin{tabular}{|c|c|c|c|}
\hline 
Amount & 1 & 2 & 5\tabularnewline
\hline 
\hline 
0 & 1 &  & \tabularnewline
\hline 
1 & 1 &  & \tabularnewline
\hline 
2 & 1 &  & \tabularnewline
\hline 
3 & 1 &  & \tabularnewline
\hline 
4 & 1 &  & \tabularnewline
\hline 
5 & 1 &  & \tabularnewline
\hline 
\end{tabular}

Based on the division of sub-problems mentioned earlier, each cell
in the table can be derived from computed cells.

For example, at cell $(1,1)$, $S(1,[1])=S(1,[])\cup S(0,[1])$. There
is no solution to $S(1,[])$ as you cannot make change with no coins,
and there is 1 solution to $S(0,[1])$ as stored in cell $(0,1)$.
Thus, the number of solution to $S(1,[1])$ is $1+0=1$. Put a $1$
at cell $(1,1)$. The first colum can be computed following the same
pattern.

2-d array: example 3

\begin{tabular}{|c|c|c|c|}
\hline 
Amount & 1 & 2 & 5\tabularnewline
\hline 
\hline 
0 & 1 & 1 & \tabularnewline
\hline 
1 & 1 & 1 & \tabularnewline
\hline 
2 & 1 &  & \tabularnewline
\hline 
3 & 1 &  & \tabularnewline
\hline 
4 & 1 &  & \tabularnewline
\hline 
5 & 1 &  & \tabularnewline
\hline 
\end{tabular}

At cell $(1,2)$, $S(1,[1,2])=S(1,[1])$. But the number solutions
to $S(1,[1])$ is already stored in cell $(1,1)$. Thus, the number
of solutions just $1$. Put a $1$ at cell $(1,2)$.

2-d array: example 4

\begin{tabular}{|c|c|c|c|}
\hline 
Amount & 1 & 2 & 5\tabularnewline
\hline 
\hline 
0 & 1 & 1 & \tabularnewline
\hline 
1 & 1 & 1 & \tabularnewline
\hline 
2 & 1 & 2 & \tabularnewline
\hline 
3 & 1 &  & \tabularnewline
\hline 
4 & 1 &  & \tabularnewline
\hline 
5 & 1 &  & \tabularnewline
\hline 
\end{tabular}

At cell $(2,2)$, $S(2,[1,2])=S(2,[1])\cup(S(0,[1,2])+[2])$. The
number solutions to $S(2,[1])$ is already stored in cell $(2,1)$,
and the number of solutions to $S(0,[1,2])$ is stored in cell $(0,2)$.
Thus, the number of solutions $1+1=2$. Put a $2$ at cell $(2,2)$.

2-d array: complete example

The complete steps for computing the table column-by-column.

Step 1: %
\begin{tabular}{|c|c|c|c|}
\hline 
Amount & 1 & 2 & 5\tabularnewline
\hline 
\hline 
0 &  &  & \tabularnewline
\hline 
1 &  &  & \tabularnewline
\hline 
2 &  &  & \tabularnewline
\hline 
3 &  &  & \tabularnewline
\hline 
4 &  &  & \tabularnewline
\hline 
5 &  &  & \tabularnewline
\hline 
\end{tabular} Step 2: %
\begin{tabular}{|c|c|c|c|}
\hline 
Amount & 1 & 2 & 5\tabularnewline
\hline 
\hline 
0 & 1 &  & \tabularnewline
\hline 
1 & 1 &  & \tabularnewline
\hline 
2 & 1 &  & \tabularnewline
\hline 
3 & 1 &  & \tabularnewline
\hline 
4 & 1 &  & \tabularnewline
\hline 
5 & 1 &  & \tabularnewline
\hline 
\end{tabular} 

Step 3: %
\begin{tabular}{|c|c|c|c|}
\hline 
Amount & 1 & 2 & 5\tabularnewline
\hline 
\hline 
0 & 1 & 1 & \tabularnewline
\hline 
1 & 1 & 1 & \tabularnewline
\hline 
2 & 1 & 2 & \tabularnewline
\hline 
3 & 1 & 2 & \tabularnewline
\hline 
4 & 1 & 3 & \tabularnewline
\hline 
5 & 1 & 3 & \tabularnewline
\hline 
\end{tabular} Step 4: %
\begin{tabular}{|c|c|c|c|}
\hline 
Amount & 1 & 2 & 5\tabularnewline
\hline 
\hline 
0 & 1 & 1 & 1\tabularnewline
\hline 
1 & 1 & 1 & 1\tabularnewline
\hline 
2 & 1 & 2 & 2\tabularnewline
\hline 
3 & 1 & 2 & 2\tabularnewline
\hline 
4 & 1 & 3 & 3\tabularnewline
\hline 
5 & 1 & 3 & 4\tabularnewline
\hline 
\end{tabular} 

2-d array: big o

The bottom-right cell stores the number of solutions to the original
problem.

What is the running time? 

What is the space (memory) requirement?

2-d array: reconstruction

\textbf{Reconstruct All Solutions}
\begin{enumerate}
\item Store back trace pointers in each cell of the table, and each pointer
points to the cell from which the current cell obtains a non-zero
value.
\item Use depth-first search starting from the bottom-right cell to reconstruct
the solutions
\item Moving left: solution skips the denomination of the current column
\item Moving up: solution uses one coin with the denomination of the current
column
\end{enumerate}
2-d array: reconstruction efficiency

What is the running time of reconstruction? How to make it more efficient?

Hint: the reconstruction can be thought of as another dynamic programming
problem.

1-d array: intro

\textbf{Improve Efficiency}

Can we do better than using $O(mn)$ space?

Yes! Because when we are working on the table one column at a time,
only the previous column is required, and we do not need the values
we have used. 

So we can store only one column, and over-write each value as we compute
the new one.

1-d array 2: example

Example: $n=5$ and $d=\{1,2,5\}$

Step 1: %
\begin{tabular}{|c|c|}
\hline 
Amount & \tabularnewline
\hline 
\hline 
0 & 1\tabularnewline
\hline 
1 & \tabularnewline
\hline 
2 & \tabularnewline
\hline 
3 & \tabularnewline
\hline 
4 & \tabularnewline
\hline 
5 & \tabularnewline
\hline 
\end{tabular} Step 2: %
\begin{tabular}{|c|c|}
\hline 
Amount & \tabularnewline
\hline 
\hline 
0 & 1\tabularnewline
\hline 
1 & 1\tabularnewline
\hline 
2 & 1\tabularnewline
\hline 
3 & 1\tabularnewline
\hline 
4 & 1\tabularnewline
\hline 
5 & 1\tabularnewline
\hline 
\end{tabular} 

Step 3: %
\begin{tabular}{|c|c|}
\hline 
Amount & \tabularnewline
\hline 
\hline 
0 & 1\tabularnewline
\hline 
1 & 1\tabularnewline
\hline 
2 & 2\tabularnewline
\hline 
3 & 2\tabularnewline
\hline 
4 & 3\tabularnewline
\hline 
5 & 3\tabularnewline
\hline 
\end{tabular} Step 4: %
\begin{tabular}{|c|c|}
\hline 
Amount & \tabularnewline
\hline 
\hline 
0 & 1\tabularnewline
\hline 
1 & 1\tabularnewline
\hline 
2 & 2\tabularnewline
\hline 
3 & 2\tabularnewline
\hline 
4 & 3\tabularnewline
\hline 
5 & 4\tabularnewline
\hline 
\end{tabular} 

1-d array: thinking

Can we store only one row? (No, why?)

How to modify the new algorithm to allow reconstruction of optimal
solutions?
\end{document}
